\section{Instability in the TPD group}

\subsection{Analysis}
The Television Personality Disorder (TPD) group changed markedly over the year. I. Petrov, one of the clinic's TPD specialists, did not take any appointments after June. In July, the average wait time for TPD patients tripled and N. Fulano and E. Medelsvensson, the other TPD specialists, saw large increases in number of patients. The average wait time for TPD patients stayed high for the rest of the year.

\begin{figure}[h]
%\centering
\makebox[0.5\textwidth][c]{\includegraphics[width = 0.65\textwidth]{plots/appts_per_month_doc_tpd2}}
\makebox[0.5\textwidth][c]{\includegraphics[width = 0.65\textwidth]{plots/delay_per_month_cond}}
\label{apmd}
\end{figure}

\subsection{Effect on the business}

Long patient wait times may lead to poor reviews and fewer patients in the future, explaining the large decrease in the number of TPD patients in August.

\begin{comment}
\begin{figure}[h]
\centering
\makebox[\textwidth][c]{\includegraphics[width = 0.7\textwidth]{plots/appts_per_month_cond}}
\label{apmc}
\end{figure}
\end{comment}

\subsection{Question}
For each of the TPD specialists, the large decreases in number of patients were closely preceded by sudden increases in patient numbers. Did the strain of increased appointments adversely affect the specialists?

\subsection{Solutions}
PMA should consider reinstating I. Petrov or hiring a new TPD specialist. If the increased patient wait times are a result of poor specialist performance due to stress, a third specialist will help relieve that stress by shouldering some of the workload.
